\documentclass{article}
\usepackage[utf8]{inputenc}
\usepackage[portuguese]{babel}

\title{Notas de Eletromagnetismo 2018/2019 \thanks{Fundamentos de Física de Halliday e Resnick}}
\author{Pedro Alexandre Gonçalves Ribeiro a85493}
\date{13 de Fevereiro de 2019}
\begin{document}
\maketitle
\tableofcontents
\newpage
\section{Introdução}
A física do eletromagnetismo foi estudada pela primeira vez pelos filósofos da Grécia antiga, que descobriram que se um pedaço de âmbar fosse friccionado e depois aproximado de pedacinhos de palha, a palha seria atraída pelo âmbar. Hoje sabemos a atração entre o âmbar e a palha se deve a uma força elétrica. Os filósofos gregos também observaram que se um tipo de pedra (um íman natural) fosse aproximado de objeto de ferro, o objeto seria atraído pela pedra. Hoje sabemos que a atração entre os ímans e os objetos de ferro se deve a uma força magnética.
% Capítulo 21
\section{Cargas Elétricas}
% Cargas Elétricas
\subsection{Cargas Elétricas}
Na verdade todos os corpos possuem muitas cargas elétricas. A carga elétrica é uma propriedade intrínseca das partículas fundamentais de que é feita a matéria; em outras palavras, é uma propriedade associada à própria existência das partículas.
\begin{itemize}
\item Quando existe igualdade (ou equilíbrio) de cargas, dizemos que o objeto é eletricamente neutro, ou seja, a carga total do objeto é zero.
\item Quando as quantidades dos dois tipos de cargas são diferentes, a carga total do objeto é diferente de zero e dizemos que o objeto está eletricamente carregado.
\end{itemize}
Os objetos eletricamente carregados interagem exercendo uma força sobre outros objetos. Para observar essa força, podemos carregar um bastão de vidro friccionado uma das extremidades com um pedaço de seda. Nos pontos de contacto entre o bastão e a seda, pequenas quantidades de carga são transferidas de um material para o outro, rompendo a neutralidade elétrica de ambos. (Friccionamos a seda no bastão para aumentar o número de pontos de contacto e, portanto, a quantidade de cargas transferidas).
% Condutores e Isoladores
\subsection{Condutores e Isoladores}
Os materiais podem ser classificados de acordo com a facilidade com a qual as cargas se movem no seu interior. Nos condutores, como cobre dos fios elétricos, o corpo humano e a água da torneira, as cargas elétricas se movem com facilidade. Nos não condutores, também conhecidos como isoladores, como os plásticos do isolamento dos fios, a borracha, o vidro e a água destilada, as cargas não se movem. Os semicondutores, como o silício e o germânio, possuem propriedades elétricas intermediárias entre as dos condutores e as dos não condutores. Os supercondutores são condutores perfeitos, materiais nos quais as cargas se movem sem encontrar nenhuma resistência.
% Lei de Coloumb
\subsection{Lei de Coloumb}
Duas partículas carregadas exercem forças uma sobre a outra. Se as cargas das partículas têm o mesmo sinal, as partículas repelem-se, ou seja, são submetidas a forças que tendem a afasta-las. Se as cargas das partículas têm sinais opostos, as partículas atraem-se, ou seja, são submetidas a forças que tendem a aproxima-las.
A esta forças de atração ou de repulsão associada à carga elétrica dos objetos chama-se \textbf{força eletrostática}. A lei que permite calcular a força exercida por \textit{partículas} carregadas é chamada de \textbf{Lei de Coloumb}.
Consideremos duas partículas carregadas (a {\textit{partícula 1}} e a {\textit{partícula 2}}). Suponhamos que a \textit{partícula 1} tem carga \textit{$q_1$} a \textit{partícula 2} tem carga \textit{$q_2$}.
\newline
A força a que está submetida a \textit{partícula 1} é dada por:
% Lei de Coloumb (Fórmula)
\newline
\[ \vec{F} = k \frac{q_1 q_2}{r^2} \vec{r}\]
\newline
em que \(\vec{r}\) é um vetor unitário na direção da reta que liga as duas partículas, \textit{r} é a distância entre as partículas e \textit{k} é uma constante. (Como qualquer vetor unitário, \(\vec{r}\) tem módulo 1 e é adimensional; a sua função é indicar uma orientação no espaço.)
Por motivos históricos (e também para simplificar outras equações), a constante eletrostática \textit{k} é escrita na forma \(\frac{1}{4 \pi \epsilon}\). Nesse caso, o módulo da força de Coloumb é:
\newline
% Corrigir (para a forma real)
\[ \vec{F} = \frac{1}{4 \pi \epsilon} \frac{\abs{|q_1|} \abs{|q_2|}}{r^2} \vec{r}\]
\newline
\subsection{A Carga é Quantizada}
Todas as cargas positivas e negativas \textit{q} são da forma:
\newline
\[q = ne,   n = \pm 1, \pm 2, \pm 3, \dots \]
\newline
em que \textit{e}, a \textbf{carga elementar}, tem o valor aproximado:
\newline
\[e = 1.602 \times 10^{-19} \hspace{0.2cm}C \]
\newline
Quando uma grandeza física pode assumir apenas certos valores, dizemos que é \textbf{quantizada}; a carga elétrica é uma dessas grandezas. É possível, por exemplo, encontrar uma partícula sem carga elétrica é uma dessas grandezas. É possível, por exemplo, encontrar uma partícula sem carga elétrica ou com uma carga de \textit{+10e} ou \textit{-6e}, mas não uma partícula com uma carga  de \textit{3.57e}.
\newpage
\subsection{A Carga é Conservada}
Quando friccionamos um bastão de vidro com um pedaço de seda, o bastão fica positivamente carregado. As medidas mostram que uma carga negativa de mesmo valor absoluto se acumula na seda. Isso sugere que o processo não cria cargas, mas apenas transfere cargas de um corpo para o outro, rompendo no processo a neutralidade de carga dos dois corpos. Esta hipótese de \textbf{conservação da carga elétrica}, proposta por Benjamin Franklin, foi comprovada exaustivamente, tanto no caso de objetos macroscópicos como no caso de átomos, núcleos e partículas elementares.
% Capítulo 22
\section{Campos Elétricos}
\subsection{O Campo Elétrico}
O campo elétrico é um \textit{campo vetorial}, uma vez que consiste numa distribuição de \textit{valores}, um para cada ponto de uma região em torno de um objeto eletricamente carregado, como uma bastão de vidro \textit{e.g}. Em princípo, podemos definir o campo elétrico num ponto das proximidades de um objeto carregado. Colocamos num ponto arbitrário \textit{P} uma carga positiva \textit{$q_0$} à qual chamamos \textit{carga de prova}, medimos a força eletrostática \(\vec{F}\) que atua sobre a carga \textit{$q_0$} e definimos o campo elétrico \(\vec{E}\) produzido pelo objeto através da seguinte equação:
\newline
\[\vec{E} \hspace{0.2cm}= \frac{\vec{F}}{q_0}\]
\newline
Assim, o módulo do campo elétrico \(\vec{E}\) no ponto \textit{P} é \(\textit{E} = \textit{F/}\textit{$q_0$}\) e a orientação de \(\vec{E}\) é a força de \(\vec{F}\) que age sobre a carga de prova (que supomos ser \textit{positiva}). Para definir o campo elétrico numa região do espaço, definimos o campo em todos os pontos da região. A unidade de campo elétrico de SI é o newton por coloumb (N/C).
\newline
Embora seja usada uma carga de prova para definir o campo elétrico produzido por um objeto carregado, o campo existe independentemente da carga de prova.
\subsection{Linhas de Campo Elétrico}
A relação entre as linhas de campo e os vetores de campo elétrico é a seguinte:
\begin{itemize}
    \item Em qualquer ponto, a orientação de uma linha de campo retiĺinea ou a orientação da tangente a uma linha de campo não retilínea é a orientação do campo elétrico \(\vec{E}\) nesse ponto;
    \item As linhas de campo são desenhadas de tal forma que o número de linhas por unidade de área, medido em plano perpendicular às linhas, é proporcional ao \textit{módulo} de \(\vec{E}\);
\end{itemize}
Assim, \textit{E} tem valores elevados nas regiões em que as linhas de campo estão mais próximas e valores pequenos nas regiões em que as linhas de campo estão mais afastadas.
\subsection{Campo Elétrico Produzido por uma Carga Pontual}
Para determinarmos o campo elétrico produzido a uma distância \textit{r} de uma carga pontual \textit{q}, colocamos uma carga de prova \textit{$q_0$} nesse ponto. De acordo com a lei de Coloumb, o módulo da força eletrostática que age sobre \textit{$q_0$} é dado por:
\newline
\[\vec{F} = \frac{1}{4\pi\epsilon} \frac{{q}{q_0}}{r^2} \vec{r}\]
\newline
A direção de \(\vec{F}\) é contrária à carga pontual caso esta seja positiva e na direção da mesma caso esta seja negativa. O vetor campo elétrico é dado por:
\newline
\[\vec{E} = \frac{\vec{F}}{q_0} = \frac{1}{4\pi\epsilon} \frac{q}{r^2} \vec{r}\]
\section{Lei de Gauss}
\subsection{Fluxo}
Suponhamos que uma espira quadrada de área A é exposta a um vento uniforme cuja velocidade é \(\vec{v}\). Seja \(\Phi\) o fluxo (volume por unidade de tempo) do ar através da espira. Esse fluxo depende do ângulo entre \(\vec{v}\) e o plano da espira. Se \(\vec{v}\) é perpendicular ao plano da espira, o fluxo \(\Phi\) é igual a vA.
Se \(\vec{v}\) é paralela ao plano da espira, o ar não passa pela espira e, portanto, \(\Phi\) é zero. Para um ângulo intermediário \(\theta\), o fluxo \(\Phi\) depende da componente de \(\vec{v}\) normal ao plano. Como essa componente é \(v \cos \theta\), o fluxo através da espira é dado por:
\newline
\[\Phi = (v \cos \theta) A\]
\newline
Precisamos de escrever a equação anterior na forma vetorial. Para isso, definimos um vetor área \(\vec{A}\) como um vetor cujo módulo é igual à área de uma superfície plana e cuja direção é perpendicular a essa superfície. Isto permite-nos escrever a equação como o produto escalar do vetor velocidade do vento \(\vec{v}\) pelo vetor área da espira \(\vec{A}\):
\newline
\[\Phi = v A \cos \theta = \vec{v} \vec{A}\]
\newline
em que \(\theta\) é o ângulo entre \(\vec{v}\) e \(\vec{A}\).
\subsection{Fluxo de um Campo Elétrico}
Consideremos um superfície gaussiana arbitrária (assimétrica) imersa num campo elétrico não uniforme. Vamos dividir a superfície em quadrados de área \(\Delta A\) suficientemente pequenos para que a curvatura local da superfície possa ser desprezada e os quadrados possam ser considerados planos. Estes elementos de área podem ser representados por vetores \(\Delta A\) cujo módulo é a área \(\Delta A\). Todos os vetores \(\Delta A\) são perpendiculares à superfície gaussiana e apontam para força da superfície.
Como os quadrados são arbitrariamente pequenos, o campo elétrico \(\vec{E}\) pode ser considerado constante no interior de cada quadrado; assim, para cada quadrado, os vetores \(\Delta \vec{A}\) e \(\vec{E}\) fazem um certo ângulo \(\theta\).
Segue a seguinte definição provisória do fluxo do campo elétrico para a superfície gaussiana é a seguinte:
\[\Phi = \sum{\vec{E}{\Delta \vec{A}}}\]

\end{document}







